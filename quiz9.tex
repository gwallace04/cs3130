\documentclass[letterpaper, 11pt]{article}
%\usepackage[hmargin = 1in, vmargin = 1in]{geometry}
\usepackage{amsmath}
\usepackage{amssymb}
\usepackage{enumitem}
\usepackage{mathrsfs}
\usepackage{tikz}
\usepackage{graphicx}
\usepackage{algorithmicx}
\usepackage{algpseudocode}
\usepackage{mathtools}
\usepackage[linguistics]{forest}
% \doublespacing
\setlength{\headheight}{14pt}
\usepackage{fancyhdr}
\pagestyle{fancy}
\rhead{Gabriel Wallace}
\lhead{Comp Sci 3130}

\newcommand{\card}{\text{Card}}
\newcommand{\N}{\mathbb{N}}
\newcommand{\R}{\mathbb{R}}
\newcommand{\Z}{\mathbb{Z}}
\newcommand{\Q}{\mathbb{Q}}

\newcommand{\inv}{^{-1}}
\newcommand{\abs}[1]{\lvert #1 \rvert}
\newcommand{\hwnumber}[1]{\medskip \noindent\textbf{#1.} \smallskip}
\newcommand{\hwnumbersec}[3]{\medskip \noindent\textbf{#1.} Section #2 \##3 \smallskip}
\newcommand{\Mod}[1]{\ \mathrm{mod}\ #1}
\newcommand{\Alg}[1]{\medskip \noindent\textbf{ALGORITHM} \( #1 \)} 
\newcommand{\To}{\textbf{ to }}

\DeclarePairedDelimiter{\ceil}{\lceil}{\rceil}

\begin{document}
\begin{center}
	{\LARGE Quiz 9}\\
\end{center}

\hwnumber{1} The smallest number with \(n\) digits is always of the form
\(10^{n-1}\). Thus the smallest product, and therefore the smallest number
digits is 
\[10^{n-1} \times 10^{n-1} = 10^{2n-2},\]
which has \(2n-1\) digits.


\hwnumber{2} The \(n\log n\) barrier is the theoretical lower bound for the
average case efficiency of any comparison based sorting algorithm. 


\hwnumber{3} If there are no equal elements in the array then the loop would
never break early, thus there would be \(n - 1\) comparisons.

\end{document}
