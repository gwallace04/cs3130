\documentclass[letterpaper, 11pt]{article}
%\usepackage[hmargin = 1in, vmargin = 1in]{geometry}
\usepackage{amsmath}
\usepackage{amssymb}
\usepackage{enumitem}
\usepackage{mathrsfs}
\usepackage{tikz}
\usepackage{graphicx}
\usepackage{algorithmicx}
\usepackage{algpseudocode}
\usepackage{mathtools}
\usepackage[linguistics]{forest}
% \doublespacing
\setlength{\headheight}{14pt}
\usepackage{fancyhdr}
\pagestyle{fancy}
\rhead{Gabriel Wallace}
\lhead{Comp Sci 3130}

\newcommand{\card}{\text{Card}}
\newcommand{\N}{\mathbb{N}}
\newcommand{\R}{\mathbb{R}}
\newcommand{\Z}{\mathbb{Z}}
\newcommand{\Q}{\mathbb{Q}}

\newcommand{\inv}{^{-1}}
\newcommand{\abs}[1]{\lvert #1 \rvert}
\newcommand{\hwnumber}[1]{\medskip \noindent\textbf{#1.} \smallskip}
\newcommand{\hwnumbersec}[3]{\medskip \noindent\textbf{#1.} Section #2 \##3 \smallskip}
\newcommand{\Mod}[1]{\ \mathrm{mod}\ #1}
\newcommand{\Alg}[1]{\medskip \noindent\textbf{ALGORITHM} \( #1 \)} 
\newcommand{\To}{\textbf{ to }}

\DeclarePairedDelimiter{\ceil}{\lceil}{\rceil}

\begin{document}
The table of the results from Part B is as follows:
\begin{center}
\begin{tabular}{rrr}
  \(N\) &   \(t\) &  Average Height \\
    \hline
  100 &   5 &   12.200000 \\
  100 &  10 &   12.700000 \\
  100 &  15 &   11.666667 \\
  500 &   5 &   17.200000 \\
  500 &  10 &   18.400000 \\
  500 &  15 &   18.466667 \\
 1000 &   5 &   21.000000 \\
 1000 &  10 &   20.500000 \\
 1000 &  15 &   21.866667 \\
\end{tabular}
\end{center}

In the code for the binary search tree class, the first check for inserting a
new node is if the value of the node to be inserted is less than the root.
Thus, duplicates would be the right child of the existing duplicate.

The theoretical efficiency of the algorithm to find the height is
\(\Theta(n)\).

\end{document}
