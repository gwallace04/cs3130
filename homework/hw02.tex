\documentclass[letterpaper, 11pt]{article}
%\usepackage[hmargin = 1in, vmargin = 1in]{geometry}
\usepackage{amsmath}
\usepackage{amssymb}
\usepackage{enumitem}
\usepackage{mathrsfs}
\usepackage{tikz}
\usepackage{graphicx}
\usepackage{algorithmicx}
\usepackage{algpseudocode}
% \doublespacing
\setlength{\headheight}{14pt}
\usepackage{fancyhdr}
\pagestyle{fancy}
\rhead{Gabriel Wallace}
\lhead{Comp Sci 3130}

\newcommand{\card}{\text{Card}}
\newcommand{\N}{\mathbb{N}}
\newcommand{\R}{\mathbb{R}}
\newcommand{\Z}{\mathbb{Z}}
\newcommand{\Q}{\mathbb{Q}}

\newcommand{\inv}{^{-1}}
\newcommand{\abs}[1]{\lvert #1 \rvert}
\newcommand{\hwnumber}[1]{\medskip \noindent\textbf{#1.} \smallskip}
\newcommand{\hwnumbersec}[3]{\medskip \noindent\textbf{#1.} Section #2 \##3 \smallskip}
\newcommand{\Mod}[1]{\ \mathrm{mod}\ #1}

\begin{document}
\begin{center}
	{\LARGE Homework 2}\\
\end{center}

\hwnumber{1} 

\begin{enumerate}[label = (\alph*)]
  \item Let \(f(n) = 3^{n + 1}\) and \(g(n) = 3^n\). Examine
   \begin{align*}
     \lim_{n \to \infty} \frac{f(n)}{g(n)} &= \lim_{n \to
                                                   \infty}\frac{3^{n+1}}{3^n}\\
                                           &= \lim_{n \to \infty} \frac{3 \cdot
                                           3^n}{3^n}\\
                                           &= \lim_{n \to \infty} 3\\
                                           &= 3
   \end{align*}
   Since 3 is a constant, then \(f(n) \in \Theta(g(n))\)

 \item Let \(f(n) = 3^{3n}\) and \(g(n) = 3^n\). Examine
   \begin{align*}
     \lim_{n \to \infty} \frac{f(n)}{g(n)} &= \lim_{n \to \infty}
                                                     \frac{3^{3n}}{3^n}\\
                                           &= \lim_{n \to
                                           \infty}\frac{(3^n)^3}{3^n} \\
                                           &= \lim_{n \to \infty} 3^{2n} \\
                                           &= \infty
    \end{align*}
    So \(f(n) \in \Omega(g(n))\).
\end{enumerate}

\hwnumber{2}

\noindent The list of functions ranked from smallest order of growth to largest is below.

\[1000, \ln(\ln n), \sqrt{\ln(n)}, \{\log_5 n, \lg n\}, (\lg n)^2,
\left(\sqrt{2}\right)^{lg n},\]
\[\{n, 1000n+3, 2^{\lg n}\}, \{n\cdot\lg n, \ln(n!)\}, \{n^2, 4^{\lg n}\},\]
\[n^3, \left(\frac{3}{2}\right)^n, 2^n, n2^n, e^n, n!, (n + 1)!\]

\newpage
\hwnumber{3}

Let \(T(n) = T(n/4) + B\), with initial condition \(T(1) = A\), and let \(n =
4^k\). We proceed with backwards substitution. 
\begin{align*}
  T(n) &= T(4^k)\\
       &= T(4^{k - 1}) + B\\
       &= T(4^{k - 2}) + B + B\\
       &\;\;\;\;\;\vdots \\
       &= T(1) + \underbrace{B + B + \dots + B}_{k \text{ times}}\\
\end{align*}

Thus, \(T(n) = T(1) + Bk = A + Bk\). Since \(n = 4^k\), then \(k = \log_4 n\)
so \(T(n) = A + B\log_4 n\).

\hwnumber{4}

Let \(x(n) = \frac{1}{2}(x(n - 1) + x(n - 2))\). Then,
\[x(n) - \frac{1}{2}x(n - 1) - \frac{1}{2}x(n - 2) = 0\]
So we have the characteristic equation:
\begin{align*}
  r^2 - \frac{1}{2}r - \frac{1}{2} &= 0\\
  2r^2 - r - 1 &= 0 \\
  (2r + 1)(r - 1) &= 0
\end{align*}

So \(r_1 = -\frac{1}{2}\) and \(r_2 = 1\). So the general solution is 
\begin{align*}
  x(n) &= \alpha_1 \left(-\frac{1}{2}\right)^n + \alpha_2(1)^n\\
       &= \alpha_1 \left(-\frac{1}{2}\right)^n + \alpha_2\\
\end{align*}

\newpage
\hwnumber{5}

\begin{enumerate}[label = (\alph*)]
  \item Let \(T(n) = 2T(\frac{n}{4}) + 1000\). So 
    \[a = 2, b = 4, d = 0\]
    Since 
    \[a = 2 > 1 = b^d\]
    then 
    \[T(n) \in \Theta\left(n^{\log_4(2)}\right)\]
  \item Let \(T(n) = 2T(\frac{n}{4}) + 1000n\). So 
    \[a = 2, b = 4, d = 1\]
    Since 
    \[a = 2 < 4 = b^d\]
    then 
    \[T(n) \in \Theta(n)\]
  \item Let \(T(n) = 2T(\frac{n}{4}) + 1000\sqrt{n}\). So 
    \[a = 2, b = 4, d = \frac{1}{2}\]
    Since 
    \[a = 2 = 2 = b^d\]
    then 
    \[T(n) \in \Theta\left(\sqrt{n}\log(n)\right)\]
  \item Let \(T(n) = 2T(\frac{n}{4}) + 1000n^2\). So 
    \[a = 2, b = 4, d = 2\]
    Since 
    \[a = 2 < 16 = b^d\]
    then 
    \[T(n) \in \Theta\left(n^2\right)\]
\end{enumerate}

\newpage
\hwnumber{6}
In both parts, we denote E and E' to differentiate the two identical
characters. 

\begin{enumerate}[label = (\alph*)]
  \item Sec 3.1 \#8 \\
    We proceed with selection sort. \\
    {
    \centering
    \begin{tabular}{l l l l l l l}
      E & X & A & M & P & L & E' \\
      A \textbar & X & E & M & P & L & E' \\
      A & E \textbar & X & M & P & L & E' \\
      A & E & E' \textbar & M & P & L & X  \\
      A & E & E' & L \textbar & P & M & X  \\
      A & E & E' & L & M \textbar & P & X  \\
      A & E & E' & L & M & P \textbar & X 
    \end{tabular}
    }

  \item Sec 3.1 \#11
    We proceed with bubble sort. \\
    {
    \centering
    \begin{tabular}{l l l l l l l}
      E & X & A & M & P & L & E' \\
      E & A & M & P & L & E' \textbar & X \\
      A & E & M & L & E' \textbar  & P & X \\
      A & E & L & E' \textbar & M & P & X \\
      A & E & E' \textbar  & L & M & P & X \\
      A & E \textbar  & E' & L & M & P & X \\
      A \textbar  & E & E' & L & M & P & X \\
    \end{tabular}
    }
\end{enumerate}

\end{document}
