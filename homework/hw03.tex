\documentclass[letterpaper, 11pt]{article}
%\usepackage[hmargin = 1in, vmargin = 1in]{geometry}
\usepackage{amsmath}
\usepackage{amssymb}
\usepackage{enumitem}
\usepackage{mathrsfs}
\usepackage{tikz}
\usepackage{graphicx}
\usepackage{algorithmicx}
\usepackage{algpseudocode}
\usepackage{mathtools}
% \doublespacing
\setlength{\headheight}{14pt}
\usepackage{fancyhdr}
\pagestyle{fancy}
\rhead{Gabriel Wallace}
\lhead{Comp Sci 3130}

\newcommand{\card}{\text{Card}}
\newcommand{\N}{\mathbb{N}}
\newcommand{\R}{\mathbb{R}}
\newcommand{\Z}{\mathbb{Z}}
\newcommand{\Q}{\mathbb{Q}}

\newcommand{\inv}{^{-1}}
\newcommand{\abs}[1]{\lvert #1 \rvert}
\newcommand{\hwnumber}[1]{\medskip \noindent\textbf{#1.} \smallskip}
\newcommand{\hwnumbersec}[3]{\medskip \noindent\textbf{#1.} Section #2 \##3 \smallskip}
\newcommand{\Mod}[1]{\ \mathrm{mod}\ #1}
\newcommand{\Alg}[1]{\medskip \noindent\textbf{ALGORITHM} \( #1 \)} 
\newcommand{\To}{\textbf{ to }}

\DeclarePairedDelimiter{\ceil}{\lceil}{\rceil}

\begin{document}
\begin{center}
	{\LARGE Homework 3}\\
\end{center}

\hwnumber{1}

We search for the substring ``ABX" in the string
``ABCDABCDABCDABXD" Successful matches are in bold and unsuccessful are in
regular text. 

\begin{center}
  \begin{tabular}{cccccccccccccccc}
    A & B & C & D & A & B & C & D & A & B & C & D & A & B & X & D \\
    \hline
    \textbf{A} & \textbf{B} & X \\
               & A \\
               & & A \\
               & & & A \\
               & & & & \textbf{A} & \textbf{B} & X \\
               & & & & & A \\
               & & & & & & A \\
               & & & & & & & A \\
               & & & & & & & & \textbf{A} & \textbf{B} & X \\
               & & & & & & & & & A \\
               & & & & & & & & & & A \\
               & & & & & & & & & & & A \\
               & & & & & & & & & & & & \textbf{A} & \textbf{B} & \textbf{X} \\
  \end{tabular}
\end{center}

We see that there are 9 successful and 12 unsuccessful matches. 

\hwnumbersec{2}{3.2}{8 (a)}

\Alg{CountSubstrings(T[0 \dots n])}
\begin{algorithmic}
  \State \(count \gets 0\)
  \For{\(i \gets 0 \To n - 1\)}
    \If {\(T[i] = `A`\)}
      \For{\(j \gets i \To n\)} 
        \If{\(T[j] = `B`\)}
          \(count \gets count + 1\)
        \EndIf
      \EndFor
    \EndIf
  \EndFor
\end{algorithmic}

\medskip

In the best case, we have a string with no A's in it, and we never
enter the second for loop. Thus, we would only iterate through the string
once, so 
\[C_B(n) \in \Theta(n) \]

In the worst case, we would have a string of just A's, so we would enter the
second loop every single time. Thus, we have

\begin{align*}
  C_W(n) &= \sum_{i=0}^{n-1} \sum_{j=i}^{n} 1 \\
         &= \sum_{i=0}^{n-1} (n - i + 1) \\
         &= (n + 1) + (n + 0) + \dots + 1 \\
         &= \frac{(n+2)(n+1)}{2}
\end{align*}
So,
\[C_W(n) \in \Theta(n^2)\]

\hwnumbersec{3}{3.4}{8}

If we have an array of \(n\) elements, then we can generate a permutation of the
array and then check if that permutation is ordered. WE will always have to
make \(n -1\) comparisons, and at worst, we'll have to check \(n!\)
permutations. Thus, we'll have to make at most \((n-1)n!\) comparisons. So the
efficiency of the worst case is 
\[C_W(n) \in O((n + 1)!)\]

\hwnumbersec{4}{4.1}{7}

\begin{center}
  \begin{tabular}{l l l l l l l}
    E \textbar & X & A & M & P & L & E' \\
    E & X \textbar & A & M & P & L & E' \\
    A & E & X \textbar & M & P & L & E' \\
    A & E & M & X \textbar & P & L & E' \\
    A & E & M & P & X \textbar & L & E' \\
    A & E & L & M & P & X \textbar & E' \\
    A & E & E' & L & M & P & X \\
  \end{tabular}
\end{center}

\hwnumbersec{5}{4.1}{11 (a)}

The array with the largest number of inversion are reverse sorted arrays, since
every \(A[i]\) would be larger than every subsequent element. Thus, they would
have
\[
  \sum_{i=0}^{n-1} (n - 1 - i) = (n - 1) + (n - 2) + \dots + 2 + 1 = \frac{n(n-1)}{2} 
\]
number of inversions.

The smallest number would be sorted arrays and they would have 0 inversions. 

\hwnumbersec{6}{4.4}{8 (a, b, c, d)}

\begin{enumerate}[label = (\alph*)]
  \item Decrease by a constant factor
  \item \[C(n) = C\left(\frac{n}{3}\right) + 2\]
    with \(C(1) = 1\) and \(n = 3^k\)
  \item 
    \begin{align*}
      C(n) &= C(3^k) \\
           &= C(3^{k-2}) + C(3^{k-1}) + 2 + 2\\
           &\;\;\;\;\;\vdots \\
           &= C(3^{k-k}) + \underbrace{2 + 2 + \dots + 2}_{k \text{ times}}\\
           &= 1 + 2k
    \end{align*}
    Since \(n = 3^k\), then \(k = \log_3 n\). So \(C(n) = 2\log_3 n + 1\).
  \item The efficiency for binary search is \(2\log_2 n + 1\). Since the base
    of the log is smaller for binary than ternary search, then ternary is more
    efficient. However, both algorithms are of the same efficiency class. 
\end{enumerate}

\hwnumbersec{7}{4.5}{2}

We find the median, and since the size of the array is 7 then 
\[k = \ceil*{7 / 2} = 4\]

We proceed with the quickselect algorithm. 

\begin{center}
  \begin{tabular}{ccccccc}
    0 & 1 & 2 & 3 & 4 & 5 & 6 \\
    \hline
    \(s\) & \(i\) \\
    \textbf{9} & 12 & 5 & 17 & 20 & 30 & 8 \\
               & \(s\) & \(i\) \\
    \textbf{9} & 5 & 12 & 17 & 20 & 30 & 8 \\
               & & \(s\)  & & & &  \(i\) \\
    \textbf{9} & 5 & 8 & 17 & 20 & 30 & 12 \\
     8 & 5 & \textbf{9} & 17 & 20 & 30 & 12 \\
  \end{tabular}
\end{center}

So \(s = 2 < 3 = k - 1\), and we partition again with the right subarray.
\begin{center}
  \begin{tabular}{ccccccc}
    0 & 1 & 2 & 3 & 4 & 5 & 6 \\
    \hline
      & & & \(s\) & \(i\) \\
      & & & \textbf{17} & 20 & 30 & 12 \\
      & & & & \(s\) & & \(i\) \\
      & & & \textbf{17} & 12 & 30 & 20 \\
      & & & 12 & \textbf{17} & 30 & 20 \\
  \end{tabular}
\end{center}

So \(s = 4 > 3 = k - 1\), and we partition again with the left subarray.
However, the left subarray is a singleton, so no swaps occur. Thus, \(s = 3 = k
- 1\) and 12 is the median. 

\hwnumbersec{8}{5.1}{6}

INSERT PICTURE HERE

\hwnumbersec{9}{5.2}{5 (a, b)}

\begin{enumerate}[label = (\alph*)]
  \item If the array is made up of all equal elements, then each partition will be
  exactly in the middle. Thus, this is the best case input.
  \item If the input is a strictly decreasing array, then there will be and
    empty subarray every time the array is partitioned. Thus, this is the worst
    case input. 
\end{enumerate}

\end{document}
