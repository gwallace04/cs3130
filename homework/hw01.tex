\documentclass[letterpaper, 11pt]{article}
%\usepackage[hmargin = 1in, vmargin = 1in]{geometry}
\usepackage{amsmath}
\usepackage{amssymb}
\usepackage{enumitem}
\usepackage{mathrsfs}
% \doublespacing
\setlength{\headheight}{14pt}
\usepackage{fancyhdr}
\pagestyle{fancy}
\rhead{Gabriel Wallace}
\lhead{Comp Sci 3130}

\newcommand{\card}{\text{Card}}
\newcommand{\N}{\mathbb{N}}
\newcommand{\R}{\mathbb{R}}
\newcommand{\Z}{\mathbb{Z}}
\newcommand{\Q}{\mathbb{Q}}

\newcommand{\inv}{^{-1}}
\newcommand{\abs}[1]{\lvert #1 \rvert}
\newcommand{\hwnumber}[3]{\noindent\textbf{#1.} Section #2 \##3}

\begin{document}
\begin{center}
	{\LARGE Homework 1}\\
\end{center}

\hwnumber{1}{1.1}{6}

\begin{enumerate}[label = (\alph*)]
  \item We find $\gcd(31415, 14142)$ by Euclid's Algorithm.
    \begin{align*}
      \gcd(31415, 14142) &= \gcd(14142, 3131)\\
                         &= \gcd(3131, 1618)\\
                         &= \gcd(1618, 1513)\\
                         &= \gcd(1513, 105)\\
                         &= \gcd(105, 43)\\
                         &= \gcd(43, 19)\\
                         &= \gcd(19, 5)\\
                         &= \gcd(5, 4)\\
                         &= \gcd(4, 1)\\
                         &= \gcd(1, 0)\\
                         &= 1
    \end{align*}

  \item We see that the number of iterations for Euclid's algorithm is 11. The
    number of iterations for the method of consecutive integers is 14142, or
    approximently 1286 times faster. 

    Writing a quick Python program to run one million simulations reveals that,
    on average, Euclid's algortihm takes about 11 iterations and the consecutive
    integers method takes about 39960 iterations, or about 3633 times faster.
\end{enumerate}

\end{document}
